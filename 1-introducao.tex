%\pagestyle{plain} 

\chapter{Introdução}
\label{cap-introducao}

\section{Motivação e Justificativa}

No contexto atual de geração de dados sem precendentes como o de produção literal e ciêntifica encontra-se a necessidade de ferramentas capazes de lidar com isso. Esses dados podem ser provindos de diversas fontes como livros, artigos ciêntificos, relatórios, paginas Web, entre outros. Como por exemplo a modo de contextualização fazendo-se uma busca rápida no Google Scholar pelo termo ``text visualization'' tem-se como retorno da busca 2.590.000 documentos até a data desta pesquisa \cite{IEEEsearch} , se fosse necessário fazer uma revisão bibliografica deste tema, exigiria muito tempo devido a necessidade de fazer uma seleção entre inúmeros documentos para descobrir quais são os mais relevantes, para isso, uma ferramenta de visualização poderia facilitar muito esta tarefa  e outras mais. 

Informações provindas destes tipos de dados tem uma alta complexidade de análise e quanto maior a base de dados mais complexa é a interpretação e mais demorado o processamento, além disso a diversidade de fontes, formatos, número de atributos associado a cada instância ou elementos do conjunto também é um fator agravante. Esses dados são considerados como multidimensionais, pois, tem vários atributos sendo que cada atributo é considerado uma dimensão que pode ser representado em um espaço m-dimensional, sendo que m é o número de atributos.

O ser humano tem a incrível capacidade de processar imagens rapidademente, interpreta-las e fazer a relação entre elas, então, transformando esses dados em imagens pode-se ter uma melhor compreensão e com mais agilidade, esta que é a principal tarefa  da área de Visualização. A efetiva compreensão dos dados depende muito da técnica ou método de visualização utilizados \cite{pagliosa2013mist}. Dentre diversas técnicas algumas são melhores para determinadas situações que outras, como por exemplo, a técnica de nuvens de palavras é eficaz em aplicações que tem por finalidade fornecer uma visualização do resumo do conteúdo de documentos \cite{pagliosa2013mist}, enquanto métodos que são baseados em estruturas hierárquicas podem permitir uma exploração mais detalhada da relação de documentos de acordo com a semelhança entre eles \cite{ward2015interactive}.

Para visualização destes dados m-dimensionais as projeções multidimensionais devem permiter mapear os dados de um espaço de alta dimensão para um espaço visual de duas ou três dimensões preservando as estruturas de visinhança entre eles. Várias metodologias propõem uma combinação de metáforas com o intuito de proporcionar um conjunto mais completo de informações em um único \textit{layout} \cite{pagliosa2013mist}.
 Embora algumas metáforas favoreçam a apresentação simultânea de informações com naturezas distintas e algumas abordagens existentes tenham de fato sido bem sucedidas na criação de \emph{layouts} compostos que proporcionam efeitos visuais significativos, ainda há uma necessidade grande de técnicas e ferramentas para realizar analise de informções em documentos.

%Visualização é a comunicação de informações usando representações gráficas para reforçar a cognição humana \cite{ward2015interactive} \cite{keim2006challenges}. Com ela dados dos mais variados tipos e tamanhos podem ser representados por imagens, textos, numeros e símbolos que facilitam o reconhecimento de informações mais rapidadamente que uma análise dos dados brutos \cite{ward2015interactive}. 


%Devido a grande quantidade de infomação gerada diarimente como textos e documentos de diversos tipos, algumas tarefas exigem a necessidade da utilização de ferramentas de visualição para facilitar tarefas desde as mais comuns até mais complexas, podendo avaliar padrões, tendências, definir qual o melhor documento a ser usado para determinada tarefa, entre outras \cite{ward2015interactive}.  

%Com o aumento da produção textual nos últimos anos, como por exemplo no banco de dados de artigos da IEEE (Institute of Electrical and Electronics Engineers), fazendo-se uma busca rápida pelo termo \emph{"visualization"}, até 2010 haviam aproximadamente 89 mil artigos científicos publicados, já no início de 2016 são mais de 159 mil \cite{IEEEsearch}, um aumento percentual de aproximadamente 79\% e também fazendo-se uma busca simples no Google Scholar com termo ``text visualization'',  tem-se de retorno da busca 2.490.000 de artigos \cite{Scholar}. Com base nestas informações ter uma ferramenta que possa trabalhar com uma quantidade grande de documentos se torna necessária, pois,  exploração de documentos para fins acadêmicos de pesquisa ou qualquer outro meio que tenha por finalidade extrair informações de interesse, como, uma revisão bibliográfica sistemática exige muito tempo devido a necessidade de fazer uma seleção entre inúmeros documentos para descobrir quais são os mais relevantes sobre um determinado assunto. 

 
Além disso, com a expanção dos dispositivos com acesso a um navegador web nos dias atuais e a padronização dos navegadores, aplicações disponíveis on-line podem ser melhores aproveitadas, pois, tem uma  tem uma área de abrangência maior podendo ser utilizada por qualquer um que tenha acesso a um navegador com conexão a internet sem a necessidade de instalar programas ou bibliotecas.
 
 Este trabalho tem como finalidade fornecer técnicas de visualização em uma ferramenta eficaz para realizar...
........

